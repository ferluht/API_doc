\section*{Введение}

Любое взаимодействие с API УСПД Waviot предполагает отправку GET-запроса на один из путей адреса \url{http://stationip:8080/}, где \textit{stationip} - ip адрес УСПД. Ответ передаётся в виде json пакета, в котором измеряемые параметры кодируются OBIS кодами. Возможные пути, параметры запросов и форматы ответов описаны в документе. Информация может дополняться по мере развития API и включения новых функций. Актуальная версия документа доступна по адресу \url{http://github.com/....}

\bigskip

\bigskip

\bigskip

\textbf{\Large{Список возможных параметров запросов}}

\bigskip

\textbf{serial} - серийный номер ПУ

\bigskip

\textbf{modemid} - идентификатор модема ПУ, может быть передан вместо параметра serial

\bigskip

\textbf{timefrom} - unix timestamp конечного времени выборки

\bigskip

\textbf{timeto} - unix timestamp начального времени выборки

\bigskip

\textbf{period} - шаг выборки в формате ISO8601, например, для 1 дня - P1D, для 1 часа PT1H. Для выборки \textbf{всех показаний}, содержащихся в истории необходимо передать нулевой интервал \textbf{PT0S}.

\bigskip

\textbf{mtype} - тип вычитываемых измерений при выгрузке текущих показаний